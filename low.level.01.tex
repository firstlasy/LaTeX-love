\documentclass{article}
\usepackage{amsfonts,amsmath,amssymb} % matematiksel ifadeler için
\usepackage[T1]{fontenc} %türkçe dil paketi
\usepackage[utf8]{inputenc} % Türkçeye has ç gibi karakterler
\usepackage[turkish]{babel} % komutların otomatik çıktıları misal contents,table,figure gibi şeyleri trye çevirir

\title{low.level.01}
\author{Murat Çoban}
\date{November 2025}

\begin{document}

\maketitle

\tableofcontents

% "%" sembolü satır içerisinde koyulduğu noktadan itibaren sağında kalan o satırdaki tüm yazıları veya kodları yorum satırına dönüştürür.

Merhaba \LaTeX %severler. \LaTeX atölyesine hoş geldiniz.

% \documentclass{x} komutu içerisine yazılan article, report, book, beamer, letter ...vb. metnin formatını belirler.

% usepackage{x} komutu içerisine yazılan amsmath, amssymb, amsfont... vb. tanımlanmış paketler LaTeX'in ana-buildinde tanımlanmamış veya daha kısa şekilde yazılsın diye hazır koların bulunduğu paket"külüphane(library)".(Kodlarınız çalışmıyorsa paketlerinizi tanımlayıp tanımlamadığınıza dikkat edin!)

% matematiksel ifadeler için kısaca amsmath, amssymb, amsfont paketlerini kullanacağız. Türkçe dil desteği için kullanılacak paketleri de en başta tanımladım. Başka paketler ekleyebilir ya da elimizde varolan paketleri çıkarabiliriz. Şimdilik bu paketler yeterli olacaktır.

% en üstteki barda ,Help kısmının sağında, dosya adı "low.level.01" dir. Bu adın değişmesi metnin başlığını değiştirmez ancak dosya adını oraya ne yazdıysanız o olarak gösterir. Yani dosya adı low.level.01 olarak gözükecek ama metnin içerisindeki ana başlık başka olabilir.

% \title{x} metninizin ismi, ana başlık, başat başlık olarak nitelendirilen kısımdır.

% \author{x} yazarların isminin yazıldığı komut.(Birden fazla yazar ismi yazabilir!)

% \date{x} metni kaleme aldığınız ya da metninizin yayınlandırğı tarih.

% \maketitle komutu yazılmazsa çıktı kısmında ana başlık, yazar ismi ve yayınlanma tarihi gözükmez!

% \section, \subsection, \subsubsection gibi komutlar metnin başlığı altında o metinde neyi anlatıyorsak onunla alakalı alt başlıklardır. \section, \subsection, \subsubsection ekledikçe metin içerisindeki sırasına göre otomatik numaralandırır.

\section{İntegral}
    \subsection{Belirli integral}
        \subsubsection{Örnekler}

% Sadece başlıklar gözüksün ama numaralar gözükmesin istiyorsanız yıldız ekleyin.

\section*{İntegral}
    \subsection*{Belirli integral}
        \subsubsection*{Örnekler}

% \tableofcontens komutu * lı olmayan sectionları ve numaralarını ve sayfa numaralarını içindekiler şeklinde en başta belirtir.

% matematiksel ifadelerin tamamı $...$  ya da \[...\] arasına yazılır.

\int^{5}_{0}(x^2+5)dx=?

% editör bu kodu tanıyor ve çıktıda istediğimiz gibi gösteriyor ama yazımın yanlış olduğu konusunda uyarıyor.Çünkü $$ arasına koymadık.Bu başta sorun gibi gösüzkmeyebilir ama birkaç deneme yaparsanız veyahut bir şeyler eklerseniz hatalı yazımın yanına sorunu anlayacaksınız. Peki nasıl yazılmalı?

$\int^{5}_{0}(x^2+5)dx=?$ 
\[\int^{5}_{0}(x^2+5)dx\] 

% elbette yukarıdaki iki şekilde!(dikkatli dimağların bu iki $...$  , \[...\] ifadenin çıktılarında ne gibi farklılıklar olduğunu bulacağına inanıyorum ;)

% satır değiştirmek için \\ kodun sonuna ekleriz. ancak bu komut sadece boşluk bırakmadan alt satıra geçirir. Bunun için \\ komutundan sonra sadece bir satır boşluk bırakıp sonra yazacağımız koda devam edebiliriz. Örnek:

\LaTeX öğreniyorum.\\
Yaşasın.

\LaTeX öğreniyorum.\\

Yaşasın.

% şimdi kaslarımızı geliştirmek için bir soru çözmeyi deneyelim. Adım adım işlemlerimizi yazalım.
\section{Bir soru}

$\int^{5}_{0}(x^2+5)dx=?$\\

$=\frac{x^3}{3}+5x\Big|^{5}_{0}$\\

$=\frac{5^3}{3}+5*5-(0)$\\

$=\frac{200}{3}$

\end{document}
